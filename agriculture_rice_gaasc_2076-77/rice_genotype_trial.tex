\documentclass[12pt,]{article}
\usepackage{lmodern}
\usepackage{setspace}
\setstretch{2}
\usepackage{amssymb,amsmath}
\usepackage{ifxetex,ifluatex}
\usepackage{fixltx2e} % provides \textsubscript
\ifnum 0\ifxetex 1\fi\ifluatex 1\fi=0 % if pdftex
  \usepackage[T1]{fontenc}
  \usepackage[utf8]{inputenc}
\else % if luatex or xelatex
  \ifxetex
    \usepackage{mathspec}
  \else
    \usepackage{fontspec}
  \fi
  \defaultfontfeatures{Ligatures=TeX,Scale=MatchLowercase}
\fi
% use upquote if available, for straight quotes in verbatim environments
\IfFileExists{upquote.sty}{\usepackage{upquote}}{}
% use microtype if available
\IfFileExists{microtype.sty}{%
\usepackage{microtype}
\UseMicrotypeSet[protrusion]{basicmath} % disable protrusion for tt fonts
}{}
\usepackage[margin=1in]{geometry}
\usepackage{hyperref}
\PassOptionsToPackage{usenames,dvipsnames}{color} % color is loaded by hyperref
\hypersetup{unicode=true,
            pdftitle={Rice genotype trial: A set-up manual},
            pdfauthor={Deependra Dhakal},
            colorlinks=true,
            linkcolor=Maroon,
            citecolor=DodgerBlue4,
            urlcolor=Blue,
            breaklinks=true}
\urlstyle{same}  % don't use monospace font for urls
\usepackage{longtable,booktabs}
\usepackage{graphicx,grffile}
\makeatletter
\def\maxwidth{\ifdim\Gin@nat@width>\linewidth\linewidth\else\Gin@nat@width\fi}
\def\maxheight{\ifdim\Gin@nat@height>\textheight\textheight\else\Gin@nat@height\fi}
\makeatother
% Scale images if necessary, so that they will not overflow the page
% margins by default, and it is still possible to overwrite the defaults
% using explicit options in \includegraphics[width, height, ...]{}
\setkeys{Gin}{width=\maxwidth,height=\maxheight,keepaspectratio}
\IfFileExists{parskip.sty}{%
\usepackage{parskip}
}{% else
\setlength{\parindent}{0pt}
\setlength{\parskip}{6pt plus 2pt minus 1pt}
}
\setlength{\emergencystretch}{3em}  % prevent overfull lines
\providecommand{\tightlist}{%
  \setlength{\itemsep}{0pt}\setlength{\parskip}{0pt}}
\setcounter{secnumdepth}{5}
% Redefines (sub)paragraphs to behave more like sections
\ifx\paragraph\undefined\else
\let\oldparagraph\paragraph
\renewcommand{\paragraph}[1]{\oldparagraph{#1}\mbox{}}
\fi
\ifx\subparagraph\undefined\else
\let\oldsubparagraph\subparagraph
\renewcommand{\subparagraph}[1]{\oldsubparagraph{#1}\mbox{}}
\fi

%%% Use protect on footnotes to avoid problems with footnotes in titles
\let\rmarkdownfootnote\footnote%
\def\footnote{\protect\rmarkdownfootnote}

%%% Change title format to be more compact
\usepackage{titling}

% Create subtitle command for use in maketitle
\providecommand{\subtitle}[1]{
  \posttitle{
    \begin{center}\large#1\end{center}
    }
}

\setlength{\droptitle}{-2em}

  \title{Rice genotype trial: A set-up manual}
    \pretitle{\vspace{\droptitle}\centering\huge}
  \posttitle{\par}
    \author{Deependra Dhakal}
    \preauthor{\centering\large\emph}
  \postauthor{\par}
      \predate{\centering\large\emph}
  \postdate{\par}
    \date{6/28/2019}

\usepackage{fancyhdr}
\pagestyle{fancy}
\fancyhead[L]{Deependra Dhakal}
\fancyhead[R]{Rice genotype trial ...}
\usepackage{lineno}
\usepackage{booktabs}
\usepackage{longtable}
\usepackage{array}
\usepackage{multirow}
\usepackage{wrapfig}
\usepackage{float}
\usepackage{colortbl}
\usepackage{pdflscape}
\usepackage{tabu}
\usepackage{threeparttable}
\usepackage{threeparttablex}
\usepackage[normalem]{ulem}
\usepackage{makecell}
\usepackage{xcolor}

\begin{document}
\maketitle

\hypertarget{background}{%
\section{Background}\label{background}}

Most important cereal for more than half of the world's population (Fageria et al., 2003), Muthayya et al. (\protect\hyperlink{ref-muthayya2014overview}{2014})

Rainfed rice systems dominate in Africa(Seck et al. \protect\hyperlink{ref-seck2012crops}{2012})

Upland rice refers to rice grown on both flat and sloping fields that are prepared and seeded under dryland conditions and depend on rainfall for moisture. This is also known as dryland, rainfed, or aerobic rice. This type of rice cultivation is most common on small- and medium-size farms in South America, Asia, and Africa. Brazil is the world's largest producer of upland rice (Fageria et al., 1982; IRRI, 1984; Fageria, 2001a).

Flooded rice is grown on flat land with controlled irrigation. It is also known as irrigated, lowland, or waterlogged rice. Lowland rice may be planted by drilling the seed into dry soil, by broadcasting pre-germinated seed into flooded fields, or by planting seedlings into flooded fields by machine or by hand. Rice planted into dry soil is commonly flooded when seedlings are 25--30 days old.

Comparison of lowland and upland rice system mentions following notable features of the two cultures:

Lowland: Reduced root zone during major part of crop growth; High tillering; Thin and
shallow root system; Stable and high yield
Upland: Oxidized root zone during major part of crop growth; Low tillering; Relatively low tillering; Vigorous and deep root system; Unstable and low yield

Semidwarf varieties are suited for lowland conditions because they lend themselves to improved cultural practices.

Oryza sativa L. and Oryza glaberrima Steud. are cultivated species of rice. Oryza sativa is widely cultivated, but O. glaberrima is mainly grown in Africa where it is rapidly being replaced by O. sativa. The two species show small morphological differences, but hybrids between them are always sterile (Chang, 1976).

Paddy, on milling, gives approximately 20\% husk, 50\% whole rice, 16\% broken rice, and 14\% bran and meal (Purseglove, 1985).

The Agriculture Development Strategy (ADS) of Government of Nepal (GoN) has also given top priority to rice for ensuring food security and enhancing economic growth. In most of the rice growing areas fertilizer use is lower than the recommended rate by the government which partially contributes to lower yields (3.3 t ha-1 national average) than potential yields (\textgreater{} 5 t ha-1).

Ghaiya dhan (Upland rice) has on average low productivity

Khus, G.S. (1984). Terminology for Rice Growing environments. International rice Research Institute. Los Banos, Philippines

Importance of rice in achiving nutritional security. Major source of calorie for Nepalese. On average, samples of rice grain contain 80\% starch, 12\% percent water, 7.5\% protein, and 0.5\% ash (Chandler 1979). Rice grain is also a good source of vitamins B and E, riboflavin, thiamine and niacin but has little or no vitamin A, C or D (Grist 1986; Juliano 1993)

\hypertarget{cultivation-practices}{%
\section{Cultivation practices}\label{cultivation-practices}}

\hypertarget{land-preparation}{%
\subsection{Land preparation}\label{land-preparation}}

\begin{itemize}
\tightlist
\item
  Plough the field when dry
\item
  Create small bunds for initial wetting of transplanting bed (\textasciitilde{}100 \(m^2\))
\item
  Allow water to infiltrate and form saturated layer
\item
  Puddle and level the field
\item
  If compost/FYM is applied, spread the compost/FYM one month prior to seedling transplanting and mix it thoroughly through tilling.
\end{itemize}

\hypertarget{seed-rate}{%
\subsection{Seed rate}\label{seed-rate}}

\begin{itemize}
\tightlist
\item
  Seed rate: 50 kg/ha for OPV and 20 kg/ha for hybrid
\item
  Adjusted seed rate for young seedling transplanting in seedling limited situation: 40 kg/ha for OPV
\end{itemize}

\hypertarget{seed-bed-preparation}{%
\subsection{Seed bed preparation}\label{seed-bed-preparation}}

\begin{itemize}
\tightlist
\item
  Wet nursery beds will be used to grow the rice seedling.
\item
  A requirement of 700 m2 area for transplanting of one hectare is set as reference.
\item
  The bed will have fine tilth by 2-3 ploughing and levelled.
\item
  In each bed, a basal dose of 8 g P2O5/ m2 area will be applied.
\item
  Rice seeds will be soaked in water (room temperature) for 8 hr inside jute sacks.
\item
  After 8 hr, the seeds will be cleaned with fresh water and will keep in shade with spraying with water and stirring for air circulation until sprouting.
\item
  The sprouted seed will be broadcasted or line sowed in wet-bed.
\item
  First few days (5 d) keep the field saturated and then increase water level to 5 cm. - A drainage channel of 30 cm will be constructed between nursery beds (bed width= 1.25 m) if area is getting excess rainfall.
\item
  After 10 days of seed sowing 10 g /m2 urea will be broadcasted.
\item
  Within 20-25 days the seedling will be transplanted in main field.
\end{itemize}

\hypertarget{transplanting-time}{%
\subsection{Transplanting time}\label{transplanting-time}}

Transplanting time: June-July

\hypertarget{planting-distance}{%
\subsection{Planting distance}\label{planting-distance}}

\begin{itemize}
\tightlist
\item
  OPV 20 x 20 cm with 2-3 seedlings/ hill
\item
  Hybrid 25 x 20 cm with 1 seedling/hill
\end{itemize}

\hypertarget{water-management}{%
\subsection{Water management}\label{water-management}}

\begin{itemize}
\tightlist
\item
  Water will be maintained at a depth of 2 cm up to panicle initiation and 5 cm thereafter up to one week before harvest.
\end{itemize}

\hypertarget{fertilizer-recommendation}{%
\subsection{Fertilizer recommendation}\label{fertilizer-recommendation}}

\begin{itemize}
\tightlist
\item
  100:30:30 kg NP2O5K2O/ha (GoN)
\item
  20 kg ZnSO4 + Borax 5 kg/ha (As per necessity)
\end{itemize}

\hypertarget{weed-control}{%
\subsection{Weed control}\label{weed-control}}

\begin{itemize}
\tightlist
\item
  Weeds were controlled by two-hand weeding at 20 and 40 d after transplanting.
\end{itemize}

\hypertarget{methodology}{%
\section{Methodology}\label{methodology}}

\hypertarget{quantity-of-rice-seed-available-for-transplanting}{%
\subsection{Quantity of rice seed available for transplanting}\label{quantity-of-rice-seed-available-for-transplanting}}

\begin{itemize}
\tightlist
\item
  100 gm of each of 14 pipeline entries
\item
  50 gm of Ghaiya-1
\end{itemize}

\hypertarget{seedling-and-field-plot-requirement-calculation}{%
\subsection{Seedling and field plot requirement calculation}\label{seedling-and-field-plot-requirement-calculation}}

\begin{itemize}
\tightlist
\item
  With the given recommendation of 40 kg/ha, 100 gm of seed suffices 25 \(m^2\) and 50 gm of seed suffices 12.5 \(m^2\).
\item
  Since the limitting amount of seed is that for check variety Ghaiya-1, each Replication will have to include 2.5 \(m^2\) strip of the check variety, ideally. The strip will ideally contain eight 20 cm spaced rows of rice seedlings (if there is limitation to constructing elongated rows; \emph{Case I}) or six 20 cm spaced rows of rice seedlings (if there is ample space to accomodate longer rows; \emph{Case II}). Thus for each case, there shall be following Net plot area specification:
\end{itemize}

For Ghaiya-1:

\begin{itemize}
\tightlist
\item
  Case I:

  \begin{itemize}
  \tightlist
  \item
    Length: 1.6
  \item
    Width: 1.6
  \end{itemize}
\item
  Case II:

  \begin{itemize}
  \tightlist
  \item
    Length: 2.1
  \item
    Width: 1.2
  \end{itemize}
\end{itemize}

For remaining 19 genotypes:

\begin{itemize}
\tightlist
\item
  Case I (16 rows):

  \begin{itemize}
  \tightlist
  \item
    Length: 1.6
  \item
    Width: 3.2
  \end{itemize}
\end{itemize}

Case II (12 rows):
- Length: 2.1
- Width: 2.4

\hypertarget{area-estimates-for-case-i}{%
\subsubsection{Area estimates for case I}\label{area-estimates-for-case-i}}

\begin{itemize}
\tightlist
\item
  Thus, Net area estimates for a replication translates to 99.84 \(m^2\), using different plot sizes for Ghaiya and rest of the other 19 genotypes.
\item
  If 0.6 m spacing is allowed between two consecutive plots, total coverage would be (in linear placement of the plots):

  \begin{itemize}
  \tightlist
  \item
    Length: 42.8 \(m\)
  \item
    Breadth: 3.2 \(m\)
  \end{itemize}
\item
  Thus, Gross area for a single replication translates to 136.96 \(m^2\)
\end{itemize}

\hypertarget{area-estimates-for-case-ii}{%
\subsubsection{Area estimates for case II}\label{area-estimates-for-case-ii}}

\begin{itemize}
\tightlist
\item
  Note that Case II requires a longer replication strip. This might be infeasible.
\item
  Note however, Gross area for a single replication translates to 126.72 \(m^2\) which is smaller larger than that for Case I. This could be due to rounding artifact (from earlier dimension assumption).
\end{itemize}

\hypertarget{being-stingy}{%
\subsubsection{Being stingy}\label{being-stingy}}

\begin{itemize}
\tightlist
\item
  But, were we to be stingy and use only a fraction of seed (half) of the pipeline genotypes to match Ghaiya's allottment, we would require Gross area of 68.48 \(m^2\) for a single replication.
\item
  Note that we avoid the case II because of lengthier Gross area for single replication requirement.
\end{itemize}

\hypertarget{being-practical}{%
\subsubsection{Being practical}\label{being-practical}}

\begin{itemize}
\tightlist
\item
  The proposed layout is to be only rigidly followed when enough land is available and if we were taking utmost pain in making things look pleasent (more or less square plots).
\item
  However, a more practically sound approach to layout would be to use 1 m wide plots accomodating 5 rows of transplants. The Gross linear length of the replicate is:

  \begin{itemize}
  \tightlist
  \item
    30.8 \(m\)
  \end{itemize}
\item
  Thus the Gross plot area (with plot spacing included) requirement reduces to:

  \begin{itemize}
  \tightlist
  \item
    154 \(m^2\)
  \end{itemize}
\end{itemize}

\hypertarget{being-practical-and-stingy}{%
\subsubsection{Being practical and stingy}\label{being-practical-and-stingy}}

\begin{itemize}
\tightlist
\item
  If we were to be stingy and use only a fraction of seed (half) of the pipeline genotypes to match Ghaiya's allottment, we would require Gross area of 77 \(m^2\) for a single replication.
\end{itemize}

Now suit your taste!

\hypertarget{total-area-requirement-e2}{%
\subsection{Total area requirement (E2)}\label{total-area-requirement-e2}}

\begin{itemize}
\tightlist
\item
  Using ``Being practical'' approach, total area required for trial in experiment E2 is 464.4 (Assuming all three spacings between replications for some leeway).
\end{itemize}

\hypertarget{fertilizer-requirement}{%
\subsection{Fertilizer requirement}\label{fertilizer-requirement}}

\begin{itemize}
\tightlist
\item
  Let us take Gross area (with plot spacing included but inter-replication space excluded) for fertilizer calculation;
\item
  One ha of field requires 30 kg P. 462 \(m^2\) area requires 3.0130435 kg of DAP.
\item
  3.0130435 kg DAP contains 0.5423478 kg N.
\item
  One ha of field requires 100 kg N. 100. 462 \(m^2\) of field requires 4.62 kg of N. After supplementing from DAP required N to be given from Urea is: 4.0776522 kg.
\item
  4.0776522 kg of N can be obtained from 8.8644612 kg Urea.
\item
  One ha of field requires 30 kg K. 462 \(m^2\) requires 2.31 kg MoP.
\end{itemize}

\hypertarget{experiment-design}{%
\subsection{Experiment design}\label{experiment-design}}

\includegraphics[width=0.95\linewidth]{rice_genotype_trial_files/figure-latex/field-layout-1}

\hypertarget{bibliography}{%
\section*{Bibliography}\label{bibliography}}
\addcontentsline{toc}{section}{Bibliography}

\hypertarget{refs}{}
\leavevmode\hypertarget{ref-muthayya2014overview}{}%
Muthayya, Sumithra, Jonathan D Sugimoto, Scott Montgomery, and Glen F Maberly. 2014. ``An Overview of Global Rice Production, Supply, Trade, and Consumption.'' \emph{Annals of the New York Academy of Sciences} 1324 (1). Wiley Online Library: 7--14.

\leavevmode\hypertarget{ref-seck2012crops}{}%
Seck, Papa Abdoulaye, Aliou Diagne, Samarendu Mohanty, and Marco CS Wopereis. 2012. ``Crops That Feed the World 7: Rice.'' \emph{Food Security} 4 (1). Springer: 7--24.


\end{document}
